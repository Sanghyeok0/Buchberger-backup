\chapter{Orders and Abstract Reduction Relations} 

\section{Foundedness Properties}

\begin{definition}\label{def:Orders}
    \lean{Preorder, LinearOrder}
    \mathlibok
    Let $r$ be a relation on $M$.
    Then $r$ is called
    \begin{enumerate}
        \item \textbf{reflexive} if $\Delta(M) \subseteq r$,
        \item \textbf{symmetric} if $r \subseteq r^{-1}$,
        \item \textbf{transitive} if $r \circ r \subseteq r$,
        \item \textbf{antisymmetric} if $r \cap r^{-1} \subseteq \Delta(M)$,
        \item \textbf{connex} if $r \cup r^{-1} = M \times M$,
        \item \textbf{irreflexive} if $\Delta(M) \cap r = \emptyset$,
        \item \textbf{strictly antisymmetric} if $r \cap r^{-1} = \emptyset$,
        \item an \textbf{equivalence relation} on $M$ if $r$ is reflexive, symmetric, and transitive,
        \item a \textbf{quasi-order} (\textbf{preorder}) on $M$ if $r$ is reflexive and transitive,
        \item a \textbf{partial order} on $M$ if $r$ is reflexive, transitive and antisymmetric,
        \item a \textbf{(linear) order} on $M$ if $r$ is a connex partial order on $M$, and
        \item a \textbf{linear quasi-order} on $M$ if $r$ is a connex quasi-order on $M$.
    \end{enumerate}
    
    % \[
    % \begin{tikzcd}[row sep=huge, column sep=small]
    % & \textbf{quasi-orders} = \textbf{preorders} \arrow[dr] \arrow[dl]& \\
    % \textbf{linear quasi-orders} \arrow[dr] & & \textbf{partial orders} \arrow[dl] \\
    % & \textbf{(linear) orders} = \textbf{total orders} &
    % \end{tikzcd}
    % \]
\end{definition}
    
\begin{definition}\label{def:well-founded}
    \lean{WellFounded}
    \mathlibok
    \uses{def:Orders}
    Let $r$ be a relation on $M$ with strict part $r_s$, and let $N \subseteq M$.
    \begin{enumerate}
        \item Then an element $a$ of $N$ is called \textit{$r$-minimal} (\textit{$r$-maximal}) in $N$ if there is no $b \in N$ with $b \, r_s \, a$ (with $a \, r_s \, b$).
        For $N = M$ the reference to $N$ is omitted.
    
        \item A \textit{strictly descending} (\textit{strictly ascending}) $r$-chain in $M$ is an infinite sequence $\{a_n\}_{n \in \mathbb{N}}$ of elements of $M$ such that $a_{n+1} \, r_s \, a_n$ (such that $a_n \, r_s \, a_{n+1}$) for all $n \in \mathbb{N}$.
    
        \item The relation $r$ is called \textbf{well-founded} (\textbf{noetherian}) if every non-empty subset $N$ of $M$ has an $r$-minimal (an $r$-maximal) element.
        $r$ is a \textbf{well-order} on $M$ if $r$ is a well-founded linear order on $M$.
    \end{enumerate}
\end{definition}

\begin{definition}[The ``Antisymmetrization'' of $M$]\label{def:eq_rel}
    \uses{def:Orders}
    \lean{Antisymmetrization}
    \mathlibok 
    Let $(M,\le)$ be a preordered set.  Define an equivalence relation
    \[
      \sim\;\colon\;M\times M\;\to\;\mathrm{Prop},
      \qquad
      a\sim b \;\iff\; \bigl(a\le b \wedge b\le a\bigr).
    \]
    We write $[a]$ for the equivalence class of $a$, and denote the quotient by
    \[
      \mathrm{Quot}(M,\sim)=\{\, [a]\mid a\in M\}.
    \]
\end{definition}

\begin{definition}\label{def:wqo}
    \lean{HasDicksonProperty}
    \leanok 
    \uses{def:Orders}
    Let $\preceq$ be a quasi-order on $M$ and let $N \subseteq M$. 
    Then a subset $B$ of $N$ is called a \textbf{Dickson basis}, or simply \textbf{basis} of $N$ w.r.t.\ $\preceq$, if for every $a \in N$ there exists some $b \in B$ with $b \preceq a$.
    \begin{enumerate}
        \item We say that $\preceq$ has the \textbf{Dickson property}, or is a \textbf{well-quasi-order}(wqo), if every subset $N$ of $M$ has a finite basis w.r.t.\ $\preceq$.
        \item A \textbf{well partial order}, or a wpo, is a wqo that is a proper ordering relation, i.e., it is antisymmetric.
    \end{enumerate}
\end{definition}

\begin{definition}[Minimal elements and min–classes]\label{def:min–classes}
    \uses{def:eq_rel}
    \lean{minClasses}
    \leanok 
    Let $N \subseteq M$. An element $a \in N$ is called \emph{minimal in $N$} if there is no $b \in N$ such that $b < a$ (where $<$ is the strict part of the quasi-order).
    We denote the set of all minimal elements of $N$ by:
    \[
      \operatorname{Minimal}(N) = \{ a \in N \mid \forall b \in N, \neg(b < a) \}
    \]
    The \emph{min–classes} of $N$ are the intersections of $N$ with the $\sim$-equivalence classes of its minimal elements:
    \[
      \operatorname{minClasses}(N) = \{ [a]_{\sim} \cap N \mid a \in \operatorname{Minimal}(N) \}
    \]
\end{definition}

\begin{lemma}\label{lem:minClasses_restrict_le_subset}
    \uses{def:min–classes}
    \lean{minClasses_restrict_le_subset}
    \leanok
    Let $N \subseteq M$ and let $a \in N$.  Consider the restricted subset
    \[
      N_{\le a} \coloneqq \{\, d \in N \mid d \le a \,\}.
    \]
    Then every min–class of $N_{\le a}$ is also a min–class of $N$, i.e.
    \[
      \operatorname{minClasses}(N_{\le a})
      \subseteq
      \operatorname{minClasses}(N).
    \]
\end{lemma}
\begin{proof}
  \leanok
  Let $[d] \in \operatorname{minClasses}(N_{\le a})$.
  By definition of $\operatorname{minClasses}$ there exists
  an element $d$ such that
  \[
    d \in N_{\le a}
    \qquad\text{and}\qquad
    \forall x \in N_{\le a},\; \neg(x < d).
  \]
  Thus $d \in N$ and $d \le a$, and for every $x \in N$ with $x \le a$
  we have $\neg(x < d)$.

  We claim that $[d] \in \operatorname{minClasses}(N)$.
  For this it suffices to show that $d \in N$ (which we already know)
  and that there is no $x \in N$ with $x < d$.

  So let $x \in N$ and assume, for a contradiction, that $x < d$.
  Then in particular $x \le d$, and since $d \le a$,
  by transitivity we obtain $x \le a$.
  Hence $x \in N_{\le a}$, and the minimality condition of $d$ in
  $N_{\le a}$ tells us that $\neg(x < d)$, contradicting $x < d$.

  Therefore no such $x$ exists, so $d$ is also minimal in $N$.
  Consequently $[d] \in \operatorname{minClasses}(N)$, and hence
  \[
    \operatorname{minClasses}(N_{\le a})
    \subseteq
    \operatorname{minClasses}(N).
  \]
\end{proof}


\begin{proposition}\label{prop:wqoEquivalent} % [Becker-Weispfenning1993] 160p Proposition 4.42
    \uses{def:wqo, def:min–classes}
    \lean{HasDicksonProperty.to_wellQuasiOrderedLE, WellQuasiOrderedLE.minClasses_finite_and_nonempty, finite_minClasses_implies_hasDicksonProperty}
    \leanok 
    Let $\preceq$ be a quasi-order on $M$ with associated equivalence relation $\sim$. 
    Then the following are equivalent:
    \begin{enumerate}
        \item $\preceq$ is a well-quasi-order.
        \item Whenever $\{a_n\}_{n \in \mathbb{N}}$ is a sequence of elements of $M$, then there exist $i < j$ with $a_i \preceq a_j$.
        \item \textit{For every nonempty subset $N$ of $M$, the number of min-classes in $N$ is finite and non-zero.}
    \end{enumerate}
\end{proposition}
\begin{proof}
  \leanok
   (i)$\Longrightarrow$(ii): Set $N = \{a_n \mid n \in \mathbb{N}\}$ and let $B$ be a finite basis of $N$. Pick $j \in \mathbb{N}$ such that $j > i$ for all $i \in \mathbb{N}$ with $a_i \in B$. 
   Then $a_{i_0} \preceq a_j$ for some $a_{i_0} \in B$, and the choice of $j$ implies $i_0 < j$.

   (ii)$\Longrightarrow$(iii): Suppose there exist infinitely many min-classes in some non-empty subset $N$ of $M$. Using the axiom of choice, we get an infinite sequence $\{a_n\}_{n \in \mathbb{N}}$ of pairwise $\sim$-inequivalent minimal elements in $N$. By our assumption (ii), $a_i \preceq a_j$ for some $i < j$. 
   From the minimality of $a_j$, we conclude that $a_j \preceq a_i$ and so $a_i \sim a_j$, a contradiction. If, on the other hand, $N$ has no minimal element, then we can produce a strictly descending $\preceq$-chain as in the proof of Proposition 4.31, contradicting (ii).

   (iii)$\Longrightarrow$(i): Let $N$ be a non-empty subset of $M$. 
   Choosing one element out of each of the finitely many min-classes, we can find a finite subset $B$ of $N$ such that each $b \in B$ is minimal, and such that every minimal $a \in N$ is $\sim$-equivalent to some $b \in B$. We claim that $B$ is a basis of $N$. Let $a \in N$. 
   Then the set
   \[
   N' = \{ d \in N \mid d \preceq a \}
   \]
   contains a minimal element $c$. It is easy to see that $c$ is minimal in $N$ too, and so $c \sim b$ for some $b \in B$. 
   We now have $b \preceq c \preceq a$ and hence $b \preceq a$.
\end{proof}

\begin{theorem}\label{thm:DicksonProperty_iff_WQO}
  \uses{prop:wqoEquivalent}
  \lean{HasDicksonProperty_iff_WellQuasiOrderedLE}
  \leanok
  Let \((M,\le)\) be a preorder. Then
  \[
    \text{\(\le\) has the Dickson property}
    \quad\Longleftrightarrow\quad
    \text{\(\le\) is a well quasi-order}.
  \]
\end{theorem}

\begin{proof}
  \leanok
  This is exactly the equivalence \((i)\Longleftrightarrow(ii)\) in Proposition
  (Becker--Weispfenning, Proposition~4.42).
  In our Lean development we prove the two implications as follows:
  \begin{itemize}
    \item \((i)\Rightarrow(ii)\) is shown by extracting a finite basis for the set
      \(N=\{a_n\mid n\in\mathbb N\}\) and choosing \(i<j\) accordingly.
    \item \((ii)\Rightarrow(i)\) is obtained by first proving \((ii)\Rightarrow(iii)\)
      (finiteness and nonemptiness of min--classes for every nonempty subset),
      and then applying \((iii)\Rightarrow(i)\).
  \end{itemize}
\end{proof}
    
% \begin{corollary}\label{cor:MinimalFiniteBasis} % [Becker-Weispfenning1993] 161p Corollary 4.43
%     \uses{prop:wqoEquivalent}
%     Let $\preceq$ be a well-partial-order on $M$.
%     Then every non-empty subset $N$ of $M$ has a \textup{\textbf{unique minimal finite basis}} $B$, i.e., a finite basis $B$ such that $B \subseteq C$ for all other bases $C$ of $N$. 
%     $B$ consists of all minimal elements of $N$.
% \end{corollary}
    
% \begin{corollary}\label{cor:wqo_is_wellFounded} % [Becker-Weispfenning1993] 161p Corollary 4.44
%     \uses{prop:wqoEquivalent}
%     Every well-quasi-order is well-founded.
% \end{corollary}
    
\begin{proposition}\label{prop:wqoAscendingSubsequence} % [Becker-Weispfenning1993] 160p Proposition 4.45
    \lean{WellQuasiOrdered.exists_monotone_subseq}
    \mathlibok
    \uses{prop:wqoEquivalent}
    Let $\preceq$ be a well- quasi-order on $M$, and let $\{a_n\}_{n \in \mathbb{N}}$ be a sequence of elements of $M$.
    Then there exists a strictly ascending sequence $\{n_i\}_{i \in \mathbb{N}}$ of natural numbers such that $a_{n_i} \preceq a_{n_j}$ for all $i < j$.
\end{proposition}
\begin{proof}
  \leanok
  We define the sequence $\{n_i\}_{i \in \mathbb{N}}$ recursively, and by simultaneous induction on $i$ we verify the following properties:
  \begin{enumerate}
      \item $a_{n_i} \preceq a_{n_{i+1}}$ for all $i \in \mathbb{N}$, and
      \item for all $i \in \mathbb{N}$, the set $\{n \in \mathbb{N} \mid a_{n_i} \preceq a_n\}$ is infinite.
  \end{enumerate}
  For $i=0$, let $\{b_1, \dots, b_k\}$ be a finite basis of the set $\{a_n \mid n \in \mathbb{N}\}$, and for each $j$ with $1 \le j \le k$, set
  \[
  B_j = \{n \in \mathbb{N} \mid b_j \preceq a_n\}.
  \]
  Then $\bigcup_{j=1}^k B_j = \mathbb{N}$ by the choice of $B$. 
  Since the union of finitely many finite sets is finite, we can find a $B_j$ which is infinite. 
  Moreover, $b_j = a_m$ for some $m \in \mathbb{N}$, and we set $n_0 = m$. 
  For $i+1$, we consider the set
  \[
  U_i = \{a_n \mid a_{n_i} \preceq a_n, n_i < n\}.
  \]
  By condition (ii) for $i$, the set $\{n \in \mathbb{N} \mid a_n \in U_i\}$ is infinite. 
  Choosing some finite basis of $U_i$, we can, as before, find an element $a_m$ in this basis such that $a_m \preceq a_n$ for infinitely many different $n \in \mathbb{N}$, and we take $n_{i+1} = m$. 
  Conditions (i) and (ii) obviously continue to hold. 
  It now follows easily from condition (i) and the transitivity of $\preceq$ that $\{n_i\}_{i \in \mathbb{N}}$ has the desired property.
\end{proof}

\section{Orderings on the Monomials in $k[x_1,\ldots,x_n]$}

\begin{lemma}\label{lem:degree_sum_le} % [Cox] 60p Lemma 8
  \lean{MonomialOrder.degree_sum_le}
  \leanok 
  Let $f, g \in k[x_1, \dots, x_n]$ be nonzero polynomials. Then:
  \begin{enumerate}
      \item $\operatorname{multideg}(fg) = \operatorname{multideg}(f) + \operatorname{multideg}(g)$.
      \item If $f + g \neq 0$, then $\operatorname{multideg}(f + g) \le \max(\operatorname{multideg}(f), \operatorname{multideg}(g))$. 
        If, in \textit{addition}, $\operatorname{multideg}(f) \neq \operatorname{multideg}(g)$, then \textit{equality occurs}.
  \end{enumerate}
\end{lemma}

\begin{lemma}\label{lem:degree_sum_le_syn}
  \lean{MonomialOrder.degree_sum_le_syn}
  \leanok 
  Let $\iota$ be an index set and $s \subset \iota$ a finite subset. For each $i \in s$, let $h_i \in k[x_1,\dots,x_n]$. 
  Then the following inequality holds:
  \[
  \operatorname{multideg}\left(\sum_{i \in s} h_i\right) \le \max_{i \in s} \left\{ \operatorname{multideg}(h_i) \right\}
  \]
  where the $\max$ is taken with respect to the monomial order.
\end{lemma}
\begin{proof}
  \leanok
  Let $M = \max_{i \in s} \{ \operatorname{multideg}(h_i) \}$. 
  Any monomial $x^b$ appearing in the sum $\sum_{i \in s} h_i$ must be a monomial in at least one of the summands, say $h_{i_0}$ for some $i_0 \in s$.
  By definition, the multidegree of any such term is bounded by the multidegree of the polynomial it belongs to, so $b \le \operatorname{multideg}(h_{i_0})$.
  Also by definition, $\operatorname{multideg}(h_{i_0}) \le M$.
  Therefore, $b \le M$ for any monomial $x^b$ in the sum. This implies that the multidegree of the sum itself cannot exceed $M$.
\end{proof}